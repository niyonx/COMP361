% --------------------------------------------------------------
% This is all preamble stuff that you don't have to worry about.
% Head down to where it says "Start here"
% --------------------------------------------------------------
 

% --------------------------------------------------------------
%                         Start here
% --------------------------------------------------------------
 
%\renewcommand{\qedsymbol}{\filledbox}

\title{TUTORIAL 8}%replace X with the appropriate number
\author{TRISTAN GLATARD\\ %replace with your name
COMP 361 Numerial Methods} %if necessary, replace with your course title
\date{November 9, 2018} 
\maketitle

\begin{exercise}{1} %You can use theorem, proposition, exercise, or reflection here. 
Numerical integration of the initial value problem
$$y^{\prime\prime} + y^\prime-y = 0, y(0) = 0, y^\prime(0) = 1$$
yielded y(1) = 0.741028. What is the value of $y^\prime(0)$ that would result in y(1) = 1, assuming that y(0) is unchanged?

\textbf{Solution.} 

Let's denote $y^\prime(0) = u$, we know that y(1) is a function of u that we denote y(1) = $\theta(u)$.

The approximation by Euler's is:
$$y(1) = y(0) + y^\prime(0)*h$$

or
\begin{align}
 \theta(u_1) = y_0 + u_1*h   
\end{align}

where h is the step size

Now the same Euler's approximation applies to the second condition
\begin{align}
 \theta(u_2) = y_0 + u_2*h   
\end{align}

Because we use the same h, (1) and (2) becomes

$$\frac{\theta(u_1)-y_0}{u_1} = \frac{\theta(u_2)-y_0}{u_2}$$

or

\begin{align}
u_2 &= u_1 \frac{\theta(u_2)-y_0}{\theta(u_1)-y_0} \notag \\   
&=1\frac{1-0}{0.741028-0} \notag \\
&=1.349477 \notag
\end{align}

\end{exercise}

%EXERCISE 2-----------------------------------------------------
\begin{exercise}{2} %You can use theorem, proposition, exercise, or reflection here.  
The solution of the differential equation
$$y^{\prime\prime\prime} + y^{\prime\prime} + 2y^{\prime} = 6$$
with the initial conditions $y(0) = 2, y^\prime(0) = 0$, and $y^{\prime\prime}(0) = 1$ yielded y(1) = 3.03765. When the solution was repeated with $y^{\prime\prime}(0) = 0$ (the other conditions
being unchanged), the result was $y(1) = 2.72318$. Determine the value of $y^{\prime\prime}(0)$ so that $y(1) = 0$.

\textbf{Solution.}

Let's denote $y^{\prime\prime}(0) = u$, we know that y(1) is a function of u that we denote y(1) = $\theta(u)$.

% We use linear interpolation


\end{exercise}


%EXERCISE 3-----------------------------------------------------
\begin{exercise}{3} %You can use theorem, proposition, exercise, or reflection here.  
Use first central difference approximations to transform the boundary value problem shown into simultaneous equations \textbf{Ay = b}.
$$y^{\prime\prime} = (2 + x)y, y(0) = 0, y^\prime(1) = 5$$

\textbf{Solution.}

First central difference approximations is given as
\begin{align}
y^\prime_i &= \frac{y_{i+1}-y_{i-1}}{2h} \notag \\
y^{\prime\prime}_i &= \frac{y_{i-1}-2y_i+y_{i+1}}{h^2} \notag
\end{align}
where $h=\frac{1}{m}$ (m is to be chosen) is step size
The problem becomes
\begin{align}
\frac{y_{i-1}-2y_i+y_{i+1}}{h^2} = (2+x_i)y_i, i=1,2,...m-1\notag \\
y_{i-1}-2y_i+y_{i+1} - {h^2}*(2+x_i)y_i = 0, i=1,2,...m-1\notag \\
y_{i-1} - (2h^2+x_ih^2+2)y_i+y_{i+1} = 0 , i=1,2,...m-1 \label{eq1} 
\end{align}

and  
\begin{align}
y_{m-1} - (2h^2+x_mh^2+2)y_m+y_{m+1} = 0 \label{eq2}
\end{align}

where $x_i = i*h$

The boundary conditions become
\begin{align}
y_0 &= 0 \label{eq4} \\ 
y^\prime(1) &= \frac{y_{m+1}-y_{m-1}}{2} = 5\notag \\
y_{m+1}&=10+y_{m-1} \label{eq3}
\end{align}

Put (\ref{eq3}) into (\ref{eq2}) yields
\begin{align}
2*y_{m-1} - (2h^2+x_mh^2+2)y_m= -10 \label{eq5}
\end{align}

Finally, we have a system of \textit{m+1} equations given by (\ref{eq1}), (\ref{eq4}) and (\ref{eq5}).

\end{exercise}


%EXERCISE 4-----------------------------------------------------
\begin{exercise}{4} %You can use theorem, proposition, exercise, or reflection here.  
Solve the given boundary value problem with the finite difference
method using \textit{m} = 5.
$$y^{\prime\prime} = (2 + x)y, y(0) = 0, y^\prime(1) = 5$$

\textbf{Solution.}

Using the system of equations from previous exercise, with $m=5$, $h =\frac{1}{m}=0.2$:
\begin{center}
    $\begin{cases} 
    y_0 =0 \\ 
    y_0 - 2.088y_1 + y_2 = 0 \\ 
    y_1 - 2.096y_2 + y_3 = 0 \\ 
    y_2 - 2.104y_3 + y_4 = 0 \\ 
    y_3 - 2.112y_4 + y_5 = 0 \\ 
    2y_4 - 2.12y_5 = -10
    \end{cases}$
\end{center}

or in matrix form
$$
 \begin{bmatrix} 
 1 & 0 & 0 & 0 & 0 & 0 \\ 
 1 & -2.088 & 1 & 0 & 0 & 0 \\ 
 0 & 1 & -2.096 & 1 & 0 & 0 \\ 
 0 & 0 & 1 & -2.104 & 1 & 0 \\ 
 0 & 0 & 0 & 1 & -2.112 & 1 \\ 
 0 & 0 & 0 & 0 & 2 & -2.12 & \\ 
 \end{bmatrix}
 \begin{bmatrix} 
 y_0 \\
 y_1 \\
 y_2 \\
 y_3 \\
 y_4 \\
 y_5 \\
 \end{bmatrix}
 = 
 \begin{bmatrix} 
 0 \\
 0 \\
 0 \\
 0 \\
 0 \\
 -10 \\
 \end{bmatrix}
$$

The solutions is
$$
y = 
 \begin{bmatrix} 
 0 \\
 1.918964 \\
 4.006797 \\
 6.479282 \\
 9.625613 \\
 13.850012 \\
 \end{bmatrix}
$$

Therefore, $y(1)=y_5 = 13.850012$
\end{exercise}


